%%NEED FOR LATER STUFF

%this is tesing the firgure

%\begin{figure}[h]
%    \centering %centers the figure
%    \includegraphics[width=0.8\textwidth]{./plymouthUniLogo2}%the graphic used
%    \caption{Example figure caption} %the actual action under figure
%    \label{fig:example} %gives id to the figure (if used like below, only returns number)
%\end{figure}

%\ref{fig:example} shows an example image.


\documentclass[12pt]{article}

\usepackage{setspace}   % For line spacing
\usepackage{helvet}     % Loads Helvetica (similar to Arial)
\renewcommand{\familydefault}{\sfdefault} % Makes Helvetica the default font

\usepackage[a4paper, margin=1in]{geometry} % Sets A4 paper size and 1-inch margins

\usepackage{pgfplots}
\usepackage{tikz}
\usepackage{amsmath}

\begin{document}
\onehalfspacing %set the spaceing to 1.5line spaceing

\begin{titlepage}
  \begin{center}
    \includegraphics[width=0.15\textwidth]{./plymouthUniLogo2}\\
    \large
    \textbf{University Of Plymouth}\\
    \vspace{0.15cm}
    \large
    \textbf{School of Engineering,\\ Computing, and Mathematics}
    
    \vspace{1.9cm}
    \Huge
    \textbf{A Co-Operation Level Editor}
    
    \vspace{4.5cm}
    \Large
    \textbf{Osbourne Laud Abraham Clark}\\
    10777267\\
    \large
   \textbf{ BSc (Hons) Computer Science}
    
    \vspace{1.5cm}
      

    \large
    Plymouth University\\
    United Kingdom\\
    $??^{th}$ April 2025
    
    
    
  \end{center}
\end{titlepage}

%%%%%%%%%%%%%%%%%%%%%%%%%%%%%%%%%%%%

\section*{Acknowledgements}
\addcontentsline{toc}{section}{Acknowledgements}

\section*{Abstract}
\addcontentsline{toc}{section}{Abstract}

\section*{Extra}
\addcontentsline{toc}{section}{Extra}
Word Count:\\
Code Link:

%%%%%%%%%%%%%%%%%%%%%%%%%%%%%%%%%%%%

\clearpage
\addcontentsline{toc}{section}{Contents}
\tableofcontents
\clearpage
\addcontentsline{toc}{section}{List of Figures}
\listoffigures
\clearpage

%%%%%%%%%%%%%%%%%%%%%%%%%%%%%%%%%%%%

\section{introduction <TODO LOOK AT SOFTWAREDESIGN DOC IN REPO>}
\subsection{Background}
this dissertation idea came about throuhg a problem identified in my second acedemic year while developing a mod for the game Co-Operation created by MINDFEAST. the issue found was the tedious nature of creating levels in general. once levels became big or contain a lot of objects they start to become a mess, and attempts to quickly change a single object can become a massive effort. there are multiple method used for finding and changeing object posistioning:
\begin{itemize}
    \item counting rows and columns in the game 
    \item crawling throuhg the file until you have found the correct object
    \item make continuous mental logs of where things have been placed 
\end{itemize}
even these with these methods, level designing and iteration becomes slow and tedious. my aim with this project is to create a tool to be used along side Co-Operation to make level creaetion easier and more fluid. 
\subsection{Objectives}
the objective of the porject is to 
\subsection{Deliverables}
%%%%%%%%%%%%%%%%%%

\section{Legal, Social, Ethical and Professional Issues}
%%%%%%%%%%%%%%%%%%

\section{Method Of Approach}

\subsection{technologies}
\subsubsection{unity}
to create this project i utilised unity for its easy to use graphics maniplulation methods. From the beginning it was clear a 3D solution was required for the project. If 2D was chosen it would not have given the same feed back or easy usablility to the user. By going 3D i could give the user a closer idea to what the levels will look like in game.   
\subsubsection{C\#}
i was required to code within C\# as unity was chosen to be developed within. this was benneficial as i am quite well versed in it.

\subsubsection{Libraries}
\subsubsection*{importing GLB objects}
i have used UnityGLTF to import glb object within to unity at runtime. this allowed for hot reloading objects without the need of re compilation of the entire unity project.\\
https://github.com/KhronosGroup/UnityGLTF
\subsubsection*{parsing YAML}
to be able to import and export file in YAML format, without haveing to create my own parser from scrtach, i have used YamlDotNet. This is a library built for C\# and so worked easily with unity. \\
https://github.com/aaubry/YamlDotNet 
\subsubsection{Co-Operation game}
\subsubsection{github}

\subsection{non-functional requirements}

\subsection{functional requirements}

\subsection{UML}
\subsubsection{classes}
\subsubsection{control flow}
%%%%%%%%%%%%%%%%%%

\section{project managment}
%%%%%%%%%%%%%%%%%%

\section{Preperation}
\subsection{previous attempts}

%%%%%%%%%%%%%%%%%%

\section{implementation}
\subsection{unity}

\subsection{YAML}
\subsubsection{importing}
\subsubsection{exporting}

\subsection{Reverse Engineering of Level Layout}
\subsubsection{object placement}
\subsubsection{object orientation}

\subsection{cacheing for faster loading}
%%%%%%%%%%%%%%%%%%

\section{Evaluation}
•	What went well and what went badly?  
•	Why was this the case?  
•	To what extent was the aspect under consideration responsible (vs. other contributing factors, e.g., your own performance).
•	Was your experience in line with what might have been expected given the body of knowledge within the literature?
•	To what extent does the above cause you to reconsider the choices that you made in relation to the given aspect?

\section{Conclusion}
It is a brief summary of the project and its achievements. Therefore, you should relist your project’s objectives and critically (and ruthlessly) evaluate whether you met the objectives
\subsection{further Work}

\end{document}