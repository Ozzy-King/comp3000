%%NEED FOR LATER STUFF

%this is tesing the firgure

%\begin{figure}[h]
%    \centering %centers the figure
%    \includegraphics[width=0.8\textwidth]{./plymouthUniLogo2}%the graphic used
%    \caption{Example figure caption} %the actual action under figure
%    \label{fig:example} %gives id to the figure (if used like below, only returns number)
%\end{figure}

%\ref{fig:example} shows an example image.


\documentclass[12pt]{article}

\usepackage{setspace}   % For line spacing
\usepackage{helvet}     % Loads Helvetica (similar to Arial)
\renewcommand{\familydefault}{\sfdefault} % Makes Helvetica the default font

\usepackage[a4paper, margin=1in]{geometry} % Sets A4 paper size and 1-inch margins

\usepackage{listings}
\usepackage{xcolor}
\newcommand{\NOTE}[1]{ \break \color{red} #1 \color{black} \break}

\definecolor{codegreen}{rgb}{0,0.6,0}
\definecolor{codegray}{rgb}{0.5,0.5,0.5}
\definecolor{codepurple}{rgb}{0.58,0,0.82}
\definecolor{backcolour}{rgb}{0.95,0.95,0.92}

\lstdefinestyle{mystyle}{
    backgroundcolor=\color{backcolour},   
    commentstyle=\color{codegreen},
    keywordstyle=\color{magenta},
    numberstyle=\tiny\color{codegray},
    stringstyle=\color{codepurple},
    basicstyle=\ttfamily\footnotesize,
    breakatwhitespace=false,         
    breaklines=true,                 
    captionpos=b,                    
    keepspaces=true,                 
    numbers=left,                    
    numbersep=5pt,                  
    showspaces=false,                
    showstringspaces=false,
    showtabs=false,                  
    tabsize=2
}

\lstset{style=mystyle}

\usepackage{pgfplots}
\usepackage{tikz}
\usepackage{amsmath}

\begin{document}
\onehalfspacing %set the spaceing to 1.5line spaceing

\begin{titlepage}
  \begin{center}
    \includegraphics[width=0.15\textwidth]{./plymouthUniLogo2}\\
    \large
    \textbf{University Of Plymouth}\\
    \vspace{0.15cm}
    \large
    \textbf{School of Engineering,\\ Computing, and Mathematics}
    
    \vspace{1.9cm}
    \Huge
    \textbf{A Co-Operation Level Editor}
    
    \vspace{4.5cm}
    \Large
    \textbf{Osbourne Laud Abraham Clark}\\
    10777267\\
    \large
   \textbf{ BSc (Hons) Computer Science}
    
    \vspace{1.5cm}
      

    \large
    Plymouth University\\
    United Kingdom\\
    $??^{th}$ April 2025
    
    
    
  \end{center}
\end{titlepage}

%%%%%%%%%%%%%%%%%%%%%%%%%%%%%%%%%%%%

\section*{Acknowledgements}
\addcontentsline{toc}{section}{Acknowledgements}

\section*{Abstract}
\addcontentsline{toc}{section}{Abstract}

\section*{Extra}
\addcontentsline{toc}{section}{Extra}
Word Count:\\
Code Link:

%%%%%%%%%%%%%%%%%%%%%%%%%%%%%%%%%%%%

\clearpage
\addcontentsline{toc}{section}{Contents}
\tableofcontents
\clearpage
\addcontentsline{toc}{section}{List of Figures}
\listoffigures
\clearpage

%%%%%%%%%%%%%%%%%%%%%%%%%%%%%%%%%%%%

\section{introduction <TODO LOOK AT SOFTWAREDESIGN DOC IN REPO>}
\subsection{Background}
this dissertation idea came about through a problem identified in my second academic year while developing a mod for the game Co-Operation created by MINDFEAST. the issue found was the tedious nature of creating levels. once levels became big or contain a lot of objects they start to become a mess, and attempts to quickly change a single object can become a massive effort. there are multiple method used for finding and changing object positioning:
\begin{itemize}
    \item counting rows and columns in the game 
    \item crawling through the file until you have found the correct object
    \item make continuous mental logs of where things have been placed 
\end{itemize}
even these with these methods, level designing and iteration becomes slow and tedious. my aim with this project is to create a visual, easy to use tool to be used along side Co-Operation making level creation more fluid. 
\subsection{Objectives}
the objective of this project is to create an easy to use visual level editing tool for both experience and inexperienced. it will be made for individuals in the Co-Operation community wanting to create levels easier. It will be aimed at people who have little to no understanding in modding and/or YAML. more experienced modders can utilise this tool to help create levels as they wont need to manually edit YAML files. it will also open up the creativity of modding to those who wouldn't normal attempt to make one.
\subsection{Deliverables}
the deliverable will be a piece of software that can display a close to 1:1 level representation, while facilitating a live tile manipulation solution to be able to edit the level directly. the users will be able to export their created levels to their respective folders, as well as import existing level files so that older levels can also be easily modified with the tool.

for object manipulation features that are to be required are CUD(Create, Update, Delete) on object with in the scene, as well as be able to import object in GLB format from the correct file location

scene manipulation is the zooming and panning of the camera. for zooming the mouse wheel is used with minimum and maximum limits to zooming, and for panning, middle mouse click is used as well as the speed changing based on zoom distance. 

when importing and exporting of levels it uses the YAML format described by the game documents. the exporting requirements are that it has the same level layout and can be played in the game, and core level features are supported for importing. this includes the main features of 2d art objects as well as 3d art objects and the structure of object definition them selves. the only feature that cant be supported without integration, of this tool into the game, would be "mapObject" with values that aren't "custom" since these rely on built in objects that aren't available. if the object uses previous described object it would use "base" instead of "mapObject". 

quality of life
quality of light features that will be added are feedback to users to update them on the current state of the editor. a key quality of life feature is the object highlighting both when the user hovers their mouse over and object and when interacting with the object through clicking and dragging. the other quality if life aspect is for making selecting objects easier and which object is going to be placed next. this is done through middle mouse click for selecting objects, and the next object being placed is done through on screen text in the bottom left.
%%%%%%%%%%%%%%%%%%

\section{Legal, Social, Ethical and Professional Issues}
%%%%%%%%%%%%%%%%%%

\section{Method Of Approach}

\subsection{technologies}
\subsubsection{unity}
to create this project i utilised unity for its easy to use graphics manipulation methods. From the beginning it was clear a 3D solution was required for the project as it would allwo for easy understanding of visuals, easy cell identification, as well as allow for more intuitive use of the software. If 2D was chosen it would not have given the same feedback methods or ease of usability to the user. By going 3D i could give the user a closer idea to what the levels will look like in game.
\subsubsection{C\#}
i was required to code within C\# as this is unity primary coding language . this was beneficial as i am well versed in it and have used it before(both unity and C\#) and so didnt have a learning curve of a new language.

\subsubsection{Libraries}
\subsubsection*{importing GLB objects}
i have chosen to use UnityGLTF to import GLB object, which are compatible with Co-Operation, into unity at runtime. this also allows for hot reloading object definitions without the need of re compilation of the entire unity project meaning further object definition defining could be performed.\\
https://github.com/KhronosGroup/UnityGLTF
\subsubsection*{parsing YAML}
YamlDotNet is used be able to import and export file in YAML format without having to write a parser from scratch. This is a library built for C\# and so worked easily with unity. this made importing and export much easier as there was no need to manually parse the file.\\
https://github.com/aaubry/YamlDotNet 
\subsubsection{Co-Operation game}
Co-Opertion will be used to to test if the level output file of the editor is valid to use in game and if the view within the editor it similar/ the same. this is very important to use in order to create this editor as its aim is to give real time visualisation of the level when creating and updating it

access was gained to this during comp2003, where the client, MindFeast, asked for a mod to be created for their game Co-Operation.
\subsubsection{GitHub}
GitHub has been used for version control and storage of the project files and code. currently only two branches have been made and have now been merged, with the later branch being made more towards the end of development. while version control wasn't heavily used, it was extensively used for transporting the code.

\subsection{functional requirements}
functional requirements include details such as object CRUD as well as map panning and the importing and exporting of levels. these are the core vital aspects for the application to work. even with the bare bones of these features the tool would have made level creation easier in both the visual aspect as well as the object placement.

\subsection{non-functional requirements}
non-functional requirements where the quality of life feature like: 
\begin{itemize}
	\item changing of pan speed based on zoom distance
	\item the on screen hints as to the current state of selected and next placements
	\item quick middle mouse click to select object for next placement
\end{itemize}

\subsection{UML}
\subsubsection{classes}
\subsubsection{control flow}
%%%%%%%%%%%%%%%%%%

\section{project managment}
%%%%%%%%%%%%%%%%%%

\section{Preparation}
there are three main aspect to preparation that i took. the first two are my previous attempts at a solution when i first witnessed the issue and attempted to come up with a fix. the third is the research of the libraries to be used with in the project. this was vital as i aimed for a full 3d application, requiring positioning information out of the YAML as well as the import of the GLB objects them selves.
\subsection {libraries in use}

\subsubsection*{importing GLB objects}
i am using UnityGLTF to import GLB object as this is a library that i have previously used and understand easily.\\
https://github.com/KhronosGroup/UnityGLTF
\subsubsection*{parsing YAML}
to be able to import and export file in YAML format i am useing YamlDotNet. in my previous attempts i only need the name and was able to make a simple parser. in this version (the third attempt), i need positional and rotation data from the object definition and so need a more robust parser. \\
https://github.com/aaubry/YamlDotNet

\subsection{previous attempts}
the previous attempts were hacks. they were quickly made in an attempt to make the editing at that point easier. only one works, which happens to be the 2\textsuperscript{nd} attempt, due to my shortcuts used used when programming win32.
\subsubsection{1\textsuperscript{st} attempt}
\NOTE{add images for the first attempt}
the first attempt used windows win32 header to create and show information to the users. it was entirely text based, only using buttons and input boxes, which no advance visual features. the point of its creation was to quickly create a application that could be used by the mod creation team in second year to help make levels easier. understandably it was never used as it was made after most of the levels had been developed and did see a vast amount bugs due to its two day creation. 
it did however attempt to tackle the issue of cell identification. by using images of all letter combinations from AA to ZZ, the application placed those images in the cell. within the game view the cells could be easily identified and then modified through the tool. the only issues with this design was that it required at least 2 windows open and the one for each cell that was getting edited and so the screen could become a mess of windows if multiple cells where getting edited at a time.
\subsubsection{2\textsuperscript{nd} attempt}
\NOTE{add images for the second attempt}
after the first attempt, while it was not used, testers gave feed back and with it came the second variation. the second attempt used a 2d pixel based engine called "olcPixelGAmeEngine". it worked through the idea that instead of the cells displaying all of its current items, it would show a colour based on if a specific item was in the cell. this was done by creating layers so each item had it own map.
this attempt did not see it implementation of easier to identify cells, instead its aim was to make level editing quicker. this was done by only needing the user to find the object in a searchable list and clicking the cell with feedback though colouring telling the user the item was now in that cell. identification could have been added through displaying of the images, but at this point i was also wanting little user setup so it was more user friendly instead of requiring images to be copied into the working folder of each mod.
\subsubsection{3\textsuperscript{rd} attempt}
the third attempt is this dissertation. its aim is to have easy identifiable cells as well as make them easily editable, merging both major features from the previous attempts into one. it was understood early on that a completely 3D solution would be best, but due to time restraint on other project this was not feasible.
%%%%%%%%%%%%%%%%%%

\section{implementation}
\subsection{unity}

\subsection{YAML}
\subsubsection{importing}
\subsubsection{exporting}

\subsection{Reverse Engineering of Level Layout}
one of the main issues faced in this project was the problem of object placement, both as placing them to form object based on object definitions, and placing them within the cells. this occurred as i did not have access to the game sources code since it it a piece of proprietary work created by MindFeast. while the company did offer to let me view their code, there was a required contract and with the stated requirements i chose not to sign. this let me keep the rights and associated freedom of my software. 
\subsubsection{object placement}
object placement was slightly confuseing at first but once thought trhouhg was quite simple. the confuseing came due to how the camera was placed in scene and working out which axis's where x and z.
\NOTE{add image of ingame and ineditor axises}
\begin{lstlisting}[language=C, caption=Python example]
Temp.transform.position += new Vector3(-objsArt.pos.x, objsArt.pos.y, -objsArt.pos.z);//position offset
\end{lstlisting}
to correct the placement and offseting of objects in object definitions and cells, the x and z values are simply reverse using the -(minus) operator
\NOTE{add image of code snippet for reverse}
\subsubsection{object orientation}

\subsection{cacheing for faster loading}
it was found through testing that when placing new objects there was a slight lag spike. this was identified as a resource loading issue as it only occurred when placing detail heavy objects. once the issue was identified it was remedied though a cacheing system that used C\# dictionary class to store already created object.

the method is as follows. when an object is about to be created it is first checked to see if it has already been created, using the files name as a look up reference in the dictionary. if it hasn't, it get imported into the scene through the GLFT importer library i am using. once in it is moved to an off screen location, this is so it cant be seen, and is then copied to be placed at the correct position where the mouse pointer.

this method has been done for both the gltf object and the images as both could cause lag spikes each time they were getting imported into the scene. by doing this, it considerably improved the speed meaning levels formed a lot quicker and are almost instant when reloading them. 
\NOTE{see if you can get timeings for both cached loading and uncached loading}
%%%%%%%%%%%%%%%%%%

\section{Evaluation}
•	What went well and what went badly?  
•	Why was this the case?  
•	To what extent was the aspect under consideration responsible (vs. other contributing factors, e.g., your own performance).
•	Was your experience in line with what might have been expected given the body of knowledge within the literature?
•	To what extent does the above cause you to reconsider the choices that you made in relation to the given aspect?

\section{Conclusion}
It is a brief summary of the project and its achievements. Therefore, you should relist your project’s objectives and critically (and ruthlessly) evaluate whether you met the objectives
\subsection{further Work}

\end{document}